\input tex2page
\input btxmac
\input texnames.sty
\input picscheme

% last change: 2017-01-16

\let\n\noindent
\let\p\verb
\let\q\scm
\def\meta[#1]{{\em #1}}
\let\c\centerline
\let\f\numberedfootnote

\def\quote{\bgroup\narrower\smallbreak}
\def\endquote{\smallbreak\egroup}

\ifx\shipout\undefined
\externaltitle{An \char`\\eval for TeX}
\fi

%\title{PicScheme}

%\title{An {\bf\tt\char`\\}eval for TeX}
\title{An {\bf\p{\eval}} for \TeX}

\c{Being an uncomplicated way of using Scheme}
\c{to specify typographically difficult elements}
\c{such as pictures in one's \TeX\ document}
\c{and have them automatically convert to appropriate
and}
\c{pleasing output in both the printed and online}
\c{versions of the document}

%\c{A TeX picture environment}
%\c{that uses Scheme as its extension language}

\medskip

\c{\ifx\shipout\UnDeFinED Download \fi
\urlh{https://github.com/ds26gte/eval4tex}{Version 20090506}}
\c{\urlh{http://ds26gte.github.io}{Dorai Sitaram}}
%\c{(PicScheme is part of the
%\urlh{tex2page-doc.html}{TeX2page} distribution)}

\bigskip

%Before we talk about the pictures, let me describe
%how to set up Scheme as an extension language for TeX.
%The mechanism can then be used for adding other
%features to TeX besides pictures.

\n I shall first describe how to set up Scheme as an
extension language for \TeX~\cite{texbook,latex}, then show how this
mechanism can add powerful features to \TeX, using
as example a high-level picture language.

\medskip

\tableofcontents

\beginchapter 1 The eval feature

Consider a \TeX\ document \p{circle.tex} with
the following contents:
\index{eval4tex.tex@\p{eval4tex.tex}}
\index{eval@\p{\eval}}

\verbwritefile picscheme-pi-value
\verbwrite{
The ratio of the circumference of a circle
to its diameter is
\eval{
;following prints pi, because cos(pi) = -1
(display (acos -1))
}.
}
\verbwritefile picscheme-done

\p+
\input eval4tex  % defines the \eval macro

The ratio of the circumference of a circle
to its diameter is
\eval{
+
\q{
;following prints pi, because cos(pi) = -1
(display (acos -1))
}
\p+
}.

\bye
+

\n  We have chosen a plain \TeX\ document for
our example, but the same approach works for
LaTeX documents too.

\p{circle.tex} first inputs the \TeX\ macro
file \p{eval4tex.tex}, which is available in the
eval4tex distribution, and which defines
a \TeX\ macro
called \p{\eval}.  This \p{\eval} macro is used in the body of
\p{circle.tex} and its argument is some Scheme code
enclosed in braces.

Now run \TeX\ on \p{circle.tex}:

\p{
tex circle
}

\n (If \p{circle.tex} were a La\TeX\ file, you would use
\p{latex circle} here.) This creates an
auxiliary file called \p{circle-Z-E.lisp}
that extracts the \p{\eval}-embedded Scheme code from \p{circle.tex}.

Now evaluate the Scheme file \p{circle-Z-E.lisp}.  For example, with
Racket, you could do

\p{
racket -f circle-Z-E.lisp
}

\n This produces an auxiliary \TeX\ file \p{circle-Z-E-1.tex}
containing an appropriate \TeX\ replacement for the
\p{\eval}'d Scheme code in \p{circle.tex}.  In general,
you will get an aux \TeX\ file
for each \p{\eval} in
the source document, and if the source document is
named \p{circle.tex}, the aux files are named
\p{circle-Z-E-|meta[n].tex}, where \p{|meta[n] = 1, 2,
|dots}.

Re-running \TeX\ on \p{circle.tex} will now insert the
aux file \p{circle-Z-E-1.tex} at the point of
the \p{\eval}, so the final DVI output,
\p{circle.dvi}, looks like:

\quote
\input picscheme-pi-value
\endquote

To summarize, we did

\p{
tex circle; racket -f circle-Z-E.lisp; tex circle
}

\n The first call to \p{tex} wrote out an aux file; the
call to \p{eval4tex} processed this aux file to produce
one or more other aux files; and the final call to
\p{tex} slurped these latter aux files into the
document.

This strategy is quite common in the \TeX\ world.  The
popular \TeX-support programs Bib\TeX\ and
MakeIndex, which generate
bibliographies and indices respectively, both operate
this way.

\beginsection \TeX2page

You can also run the program \TeX2page~\cite{tex2page} on \p{circle.tex} to get an HTML
version of the document with the same content.  As \TeX2page is written in
Scheme, it will automatically convert the \p{\eval}s in \p{circle.tex}
without requiring you to call Scheme separately.

\p{
tex2page circle
}

\n N.B.  \TeX\ documents intended to be processed by \TeX2page typically load the
macro file \p{tex2page.tex}.  Please see the \TeX2page documentation for
more details.

\beginchapter 2 Using eval

\index{eval@\p{\eval}}

Everything of course depends on what the Scheme code
inside the \p{\eval}s is.  As above, one can use Scheme
as a scratchpad to perform calculations that are beyond
\TeX.  But one can also use \p{\eval} to define a
specification language in Scheme, and then make
effective use of that language in subsequent
\p{\eval}s.

In the following, we use a {\em picture language}
called PicScheme, which is defined in Scheme.  The
operators of PicScheme generate \TeX\ and
MetaPost~\cite{metapost,mfbook} code fragments, but
their true power lies in their allowing the use of the
high-level language Scheme to compose complex pictures.
The PicScheme code is defined inside an \p{\eval} in
the \TeX\ macro file \p{picscheme.tex}.  For a sample use
of PicScheme, create a file \p{circlepic.tex} with
the following contents:

\verbwritefile picscheme-pi-pictured
\verbwrite{
\eval{
(set! *unit-length* (dimen 4 'cm))

(picture
  (lambda ()
    ;draw circle of dia 1 with center at (0,0)
    (draw (full-circle))
    ;draw horizontal diameter
    (draw (path (point -.5 0) '-- (point .5 0)))
    ;write length of diameter just above center
    (label (point 0 .05) "1")
    ;write length of circumference just outside
    ;circle, at 45 degrees
    (let ((x (/ .55 (sqrt 2))))
      (label (point x x) "$\\pi$"))))
}}
\verbwritefile picscheme-done

\p{
\input eval4tex     % load \eval
\input picscheme    % load picture language PicScheme
}
\scminput picscheme-pi-pictured
\p{
\bye
}

\n Note that this example contains two \p{\eval}s:
The first is the \p{\eval} in \p{picscheme.tex}
which defines the PicScheme operators, and the second
is the \p{\eval} in the body of \p{circlepic.tex}
that {\em uses} the PicScheme operators to draw the circle.

We enable the \p{\eval} Scheme code in
\p{circlepic.tex} the same way as before.  We run
\TeX\ on \p{circlepic.tex}, then Scheme on
\p{circlepic-Z-E.lisp}, and then \TeX\ again on
\p{circlepic.tex}:

\p{
tex circlepic
racket -f circlepic-Z-E.lisp
tex circlepic
}

\n The resulting \p{circlepic.dvi} has the
following picture:

\quote
\input picscheme-pi-pictured
\endquote

\n In addition, you can run \TeX2page directly on
\p{circlepic.tex} to produce a Web-browsable
version:

\p{
tex2page circlepic
}

We will now describe in detail the capabilities
provided by PicScheme.

\beginchapter 3 Drawing \TeX\ pictures with PicScheme

\index{PicScheme}
\index{picscheme.tex@\p{picscheme.tex}}

PicScheme, or the Scheme code in \p{picscheme.tex},
provides a set of picture-making primitives that is
similar in spirit to La\TeX's \p{{picture}}
environment~\cite[ch.~7]{latex}, except that the
commands are in Scheme and can be combined using full
Scheme.

You can specify {\em picture objects} such as
paths and texts and position
them in a composite picture object called a {\em picture}.
In general, a picture specification looks as follows:

\p+
\eval{
+
\q{
...
;Scheme code, enhanced with
;the picture-drawing primitives
...
}
\p+
}
+

\n The picture produced is a {\em box}, which means it
will be typeset as a giant letter might be.
We can use the usual \TeX\ approaches to centering and
displaying boxes, eg, \p{\centerline}.

\index{unit-length@\q{*unit-length*}}

The picture description makes use of a {\em unit
length}, which is stored in the variable \q{*unit-length*}
and is by default 1 bp (big point), which is 1/72 of an inch.
Regardless of the \q{*unit-length*}, PicScheme
internally manipulates lengths as multiples of big
points, and thus the default initial value of
of \q{*unit-length*} is the {\em number} 1.

\index{dimen@\q{dimen}}

In order to allow the resetting of \q{*unit-length*} easily,
PicScheme provides the procedure \q{dimen}, which converts
lengths in other units to scaled points.  For example,

\q{
(dimen 1 'in)
}

\n returns the bp equivalent of 1 inch, which is 72.
\q{dimen}'s second argument is thus a symbol naming the
unit converted from, and this can be one of \q{pt}
(point), \q{pc} (pica), \q{in} (inch), \q{bp} (big
point), \q{cm} (centimeter), \q{mm} (millimeter),
\q{dd} (didot point), \q{cc} (cicero), and \q{sp}
(scaled
point)~\cite[ch.~10]{texbook},~\cite[ch.~11]{mfbook}.
(The word {\em dimen} is merely standard \TeX\ jargon for
(non-springy) length --- the only physical dimension
that is relevant to typesetting.)

We can thus
set \q{*unit-length*} to be 42 mm by

\q{
(set! *unit-length* (* 42 (/ 25.4) 72))
}

\n where we make use of the equivalences 1 mm = 1/25.4 inch,
1 inch = 72 bp, or, we could say

\q{
(set! *unit-length* (dimen 42 'mm))
}

\n Use whichever is easier for you.  Changing
\q{*unit-length*} is a convenient way to scale a
picture without rewriting its innards.

\index{pencil-width@\q{pencil-width}}

An important length quantity is
the {\em thickness} of (the lead of) the imaginary pencil used to
draw a PicScheme picture.  By default this is 0.5 bp.
You may reset it using the \p{pencil-width} procedure.
Thus

\q{
(pencil-width 1/72)
}

\n sets the pencil width to 1/72 of whatever the
prevailing \q{*unit-length*} is.  (Setting pencil width is
probably the only occasion where it may be more
convenient to specify a true length in bp rather than a
multiple of the prevailing \q{*unit-length*}.  To set
it to 0.5 bp, you can use
\q{(pencil-width (/ (dimen .5 'bp) *unit-length*))},
which does the correct thing regardless of the value of
\q{*unit-length*}.)

\index{thin-lines@\q{thin-lines}}
\index{thick-lines@\q{thick-lines}}

The 0-argument procedures
\q{thin-lines} and \q{thick-lines} can be used to toggle between
the default pencil width and a value that is twice the
default width.
\q{(thin-lines)} sets pencil
thickness to the default .5 bp, while
\q{(thick-lines)} sets it to 1 bp.

\index{picture@\q{picture}}

A picture is introduced by a call to the
\q{picture} procedure:

\q{
(picture th)
}

\n
\q{th} is a thunk (zero-argument procedure) that
will contain commands to populate the picture.
For example,

\index{label@\q{label}}
\index{point@\q{point}}

\verbwritefile picscheme-nothing-special
\verbwrite{
\eval{
(set! *unit-length* (dimen 1 'cm))

(picture
  (lambda ()
   (label (point 0 0) "nothing special")))
}}
\verbwritefile picscheme-done

\p+
\centerline{
+
\scminput picscheme-nothing-special
\p+
}
+

\n produces

\quote
\c{\input picscheme-nothing-special }
\endquote

The only ``picture'' here is a text object, created by
the procedure \q{label}, but it
nevertheless  serves to illustrate the use of
coordinates.  The procedure \q{point} makes a point
from its $x$- and $y$-coordinates.  PicScheme represents
fully determined points as complex numbers, with the $x$-coordinate
as the real part and the $y$-coordinate as the imaginary
part, so \q{(point 2 3)} can be more succinctly entered
as \q{2+3i}.  In particular, \q{(point 0 0)} is simply
\q{0}.  In the following, we will use the shorter
complex-number notation when feasible.

The first argument to \q{label} is
the
point at which to place
the second argument, which can be any valid \TeX\
fragment.
Eg\f{Henceforth, the \p{\centerline{...}} wrapper around the
Scheme code is understood.}

\verbwritefile picscheme-news
\verbwrite{
\eval{
(picture
  (lambda ()
    (label -3 "West")
    (label +3i "North")
    (label 3 "East")
    (label -3i "South")))
}}
\verbwritefile picscheme-done
\scminput picscheme-news

\n produces

\quote
\c{\input picscheme-news }
\endquote

\index{label@\q{label}!positioning}

\n Note that the center of the text coincides with the
point supplied.
\q{label} can take a third argument that can place
the text at a different relation to the chosen point.

In the following example, the \q{label}'s point is
the center of the ``cross hairs'', and we use the
position indicators \q{ulft} and \q{lrt}
to put the text \q{"UPPER LEFT"} on the upper left of the
point,
and the  text \q{"LOWER RIGHT"}
on its lower right.
The various position
possibilities are \q{lft}, \q{ulft}, \q{top},
\q{urt}, \q{rt}, \q{lrt}, \q{bot}, \q{llft}.

\index{path@\q{path}}
\index{draw@\q{draw}}

\verbwritefile picscheme-ulft-lrt
\verbwrite{
\eval{
(picture
  (lambda ()
    (draw (path -2.5i '-- +2.5i))
    (draw (path -2.5 '-- 2.5))
    ;
    (label 0 "UPPER LEFT" 'ulft)
    (label 0 "LOWER RIGHT" 'lrt)))
}}
\verbwritefile picscheme-done
\scminput picscheme-ulft-lrt

\n produces

\quote
\c{\input picscheme-ulft-lrt }
\endquote

\index{path@\q{path}!--@\q{--}}
\index{path@\q{path}!**@\q{**}}
\index{path@\q{path}!direction@\q{direction}}
\index{path@\q{path}!controls@\q{controls}}
\index{path@\q{path}!cycle@\q{cycle}}

\n The procedure \q{path} returns a path
passing through a sequence of points.
The form of the path between successive points is
governed by the symbols \q{--} and \q{**}, which produce straight-line
and curved segments respectively.  Additional arguments such
as \q{direction}, \q{controls} (with \q{and}), and \q{cycle}
can further alter the path.
The procedure
\q{draw} draws its argument path.

\verbwritefile picscheme-path-ex
\verbwrite{
\eval{
(set! *unit-length* (dimen 1 'cm))

(picture
  (lambda ()
    (draw (path 0 '-- 10 '-- 10+10i '-- +10i '-- 'cycle))
    (let ((lft-control-point 3+12i)
          (rt-control-point 7+12i))
      (draw (path 2+2i '**
                  'controls lft-control-point
                  '** 7+2i))
      (draw (path 3+2i '**
                  'controls rt-control-point
                  '** 8+2i)))
    (label 9.8+0.2i
           "{\\it 020203120702, 030207120802}"
           'ulft)))
}}
\verbwritefile picscheme-done
\scminput picscheme-path-ex

\n produces

\quote
\c{\input picscheme-path-ex }
\endquote

\n \q{'controls z1}
uses \q{z1} as a control point (instead of having the path pass through it) for
the enclosing points.
For instance, a Bezier curve with the three points
\q{z1}, \q{zc}, \q{z2}
connects \q{z1} and \q{z2} using
\q{zc} as a control point, and can be represented as

\q{
(path z1 '** 'controls zc '** z2)
}

\q{'controls z1 'and z2} uses both \q{z1} and \q{z2} as control points
for the enclosing points.
\q{cycle}, if it occurs at all, should be the
last argument, and returns a closed path.  \q{direction}
governs the direction of the part of the path at
the adjacent point (which could be directly to the left or to the right).

\n Here is a {\em grid}:

\verbwritefile picscheme-grid
\verbwrite{
\eval{
(set! *unit-length* (dimen .5 'cm))

(picture
  (lambda ()
    ;draw vertical lines
    (let loop ((x 0))
      (when (<= x 20)
        ((if (= (modulo x 5) 0)
             thick-lines thin-lines))
        (draw (path (point x 0) '-- (point x 10)))
        (loop (+ x 1))))
    ;draw horizontal lines
    (let loop ((y 0))
      (when (<= y 10)
        ((if (= (modulo y 5) 0)
             thick-lines thin-lines))
        (draw (path (point 0 y) '-- (point 20 y)))
        (loop (+ y 1))))
    (thin-lines)))
}}
\verbwritefile picscheme-done
\scminput picscheme-grid

\n produces

\quote
\c{\input picscheme-grid }
\endquote

\index{draw-arrow@\q{draw-arrow}}
\index{draw-double-arrow@\q{draw-double-arrow}}

\q{draw-arrow} is similar to
\q{draw} except it attaches an arrow to the end of
the path.  \q{draw-double-arrow} attaches arrows to
both the beginning and end of the path.

\index{quarter-circle@\q{quarter-circle}}
\index{half-circle@\q{half-circle}}
\index{full-circle@\q{full-circle}}
\index{unit-square@\q{unit-square}}

The thunks \q{quarter-circle}, \q{half-circle} and
\q{full-circle} return the relevant portions of a
circle of diameter unity and center at the origin.
\q{unit-square} returns a unit square with its lower
left at the origin.

\index{shift@\q{shift}}
\index{rotate@\q{rotate}}
\index{rotate-about@\q{rotate-about}}
\index{reflect-about@\q{reflect-about}}
\index{slant@\q{slant}}
\index{scale@\q{scale}}
\index{x-scale@\q{x-scale}}
\index{y-scale@\q{y-scale}}
\index{z-scale@\q{z-scale}}
\index{transformation}

Being so pegged to the origin is not a problem, as
one can {\em transform} paths using
the procedures
\q{shift}, \q{rotate},
\q{rotate-about},
\q{reflect-about},
\q{slant},
\q{scale},
\q{x-scale},
\q{y-scale}, and
\q{z-scale}~\cite[ch.~15]{mfbook}.

\index{x-part@\q{x-part}}
\index{y-part@\q{y-part}}

\verbwritefile picscheme-square-peg
\verbwrite{
\eval{
(define rectangle
  (lambda (z1 z4)
    (let ((x1 (x-part z1))
          (y1 (y-part z1))
          (x4 (x-part z4))
          (y4 (y-part z4)))
      (path z1 '-- (point x4 y1) '-- z4
            '-- (point x1 y4) '-- 'cycle))))

(picture
  (lambda ()
    (label 0 "\\vbox{\\hbox{square}\\hbox{peg}}" 'llft)
    (label 4+3i "\\vbox{\\hbox{round}\\hbox{hole}}" 'urt)
    ;draw circle of dia = 8
    (draw (scale 8 (full-circle)))
    ;for square to fit in circle,
    ;its side must be 8/(sqrt 2)
    (let* ((s (/ 8 (sqrt 2)))
           (s/2 (/ s 2)))
      (draw (rectangle (point (- s/2) (- s/2))
                       (point s/2 s/2))))))
}}
\verbwritefile picscheme-done
\scminput picscheme-square-peg

\n produces

\quote
\c{\input picscheme-square-peg }
\endquote

The procedures \q{x-part} and \q{y-part} return the
$x$- and $y$-coordinate respectively of their argument
point.

In the following program for a spoked wheel,
the path for the spoke is created only once,
but we draw several differently rotated
versions of it.

\verbwritefile picscheme-spokes
\verbwrite{
\eval{
(set! *unit-length* (dimen 1 'cm))

(picture
  (lambda ()
    ;draw two concentric circles for wheel rim
    (let ((c (full-circle)))
      (draw (scale 4 c))
      (draw (scale 4.5 c)))
    (let ((one-deg (let ((pi (acos -1)))
                     (/ pi 180))))
      ;one-deg is 1 degree in radians
      (let ((spoke (path 0 '-- 2)))
        ;draw spoke at 20-deg increments
        (let loop ((i 0))
          (unless (>= i 360)
            (draw (rotate (* i one-deg) spoke))
            (loop (+ i 20))))))))
}}
\verbwritefile picscheme-done
\scminput picscheme-spokes

\n produces

\quote
\c{\input picscheme-spokes }
\endquote

\beginchapter 4 References

\bibliographystyle{plain}
\bibliography{../tex2page/tex2page}

\beginchapter 5 Index

\inputindex

\bye
